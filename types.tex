\subsection{Index types}

We have defined an axis as a pair $\nset{ax}{I}$, where $\name{ax}$ is a name and $I$ is a set, usually $[n]$ for some $n$. In this section, we consider some other possibilities for~$I$.

\subsubsection{Non-integral types}

The sets $I$ don't have to contain integers. For example, if $V$ is the vocabulary of a natural language ($V = \{ \name{cat}, \name{dog}, \ldots \}$), we could define a matrix of word embeddings:
\begin{align*}
  E &\in \mathbb{R}^{\nset{vocab}{V} \times \nset{emb}{d}}.
\end{align*}

\subsubsection{Integers with units}

If $\name{u}$ is a symbol and $n > 0$, define $[n]\name{u} = \{1\name{u}, 2\name{u}, \ldots, n\name{u}\}$. You could think of $\name{u}$ as analogous to a physical unit, like kilograms. The elements of $[n]\name{u}$ can be added and subtracted like integers ($a\name{u} + b\name{u} = (a+b)\name{u}$) or multiplied by unitless integers ($c \cdot a\name{u} = (c \cdot a) \name{u}$), but numbers with different units are different ($a \name{u} \neq a \name{v}$).

Then the set $[n]\name{u}$ could be used as an index set, which would prevent the axis from being aligned with another axis that uses different units. For example, if we want to define a tensor representing an image, we might write
\[ A \in \mathbb{R}^{\nset{height}{[h]\name{pixels}} \times \nset{width}{[w]\name{pixels}}}. \]
If we have another tensor representing a go board, we might write
\[ B \in \mathbb{R}^{\nset{height}{[n]\name{points}} \times \nset{width}{[n]\name{points}}}, \]
and even if it happens that $h = w = n$, it would be incorrect to write $A+B$ because the units do not match.

\subsubsection{Tuples of integers}

An index set could also be $[m] \times [n]$, which would be a way of sneaking ordered indices into named tensors, useful for matrix operations. For example, instead of defining an $\text{inv}$ operator that takes two subscripts, we could write
\begin{align*}
  A &\in \mathbb{R}^{\nset{ax}{{m\times n}}} = \mathbb{R}^{\nset{ax}{{[m]\times [n]}}} \\
  B &= \nfun{ax}{inv} A.
\end{align*}
We could also define an operator $\circ$ for matrix-matrix and matrix-vector multiplication:
\begin{align*}
  c &\in \mathbb{R}^{\nset{ax}{n}} \\
  D &= A \nbin{ax}{\circ} B \nbin{ax}{\circ} c.
\end{align*}
